\documentclass[]{article}
\usepackage{ctex}
\usepackage{amsmath}
%opening
\title{}
\author{}

\begin{document}

\section{}
\noindent 唯一性:设$g_1,g_2\in L^1(\nu)$均满足条件,则$$\int_{E}g_1\mathrm{d}\nu=\int_{E}g_2\mathrm{d}\nu,\forall E\in \mathcal{M}$$ 从而$g_1=g_2$ $\nu$几乎处处成立,即$g$唯一.\\
存在性:令测度$\nu'$定义如下:$$\nu'(E)=\int_{E}f\mathrm{d}\mu,$$显然$\nu'\ll \mu$,故$\nu'\ll \nu$,从而存在$g\in L^1(\nu)$使得$$\nu'(E)=\int_{E}g\mathrm{d}\nu,\forall E\in \mathcal{M},$$故$$\int_{E}f\mathrm{d}\mu=\int_{E}g\mathrm{d}\nu,\forall E\in \mathcal{M}$$
$$
\end{document}
